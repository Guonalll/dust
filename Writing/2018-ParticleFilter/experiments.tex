% !TEX root = ParticleFilter.tex
\section{Experiments\label{experiments}}

\subsection{Experiments with Uncertainty}

Purpose here is basically to see how the particle filter behaves when we give it:

\begin{enumerate}
\item Number of particles (i.e. how many does it need to work reasonably well?)
\item Randomness in particles
\item Measurement noise (external)
\item Internal randomness (e.g. in agent behaviour)
\item (Simultaneous combinations of different randomness)
\end{enumerate}

\subsection{Experiments with Measurement Noise}

\begin{enumerate}
\item Reduce the amount of information given to the particle filter (e.g. only allow it to optimise half of the state vector).
\item Aggregate the measurements (e.g. counts per area rather than individual traces).
\end{enumerate}

